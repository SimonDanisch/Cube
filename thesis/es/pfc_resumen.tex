%
% Resumen del proyecto de fin de carrera
%

\section*{Resumen:}

El render de nubes de puntos ha adquirido un renovado inter�s en los �ltimos a�os con la popularizaci�n y proliferaci�n de nuevos sistemas de adquisici�n de datos de coste medio/alto, habitualmente basados en la tecnolog�a LiDAR de escaneo l�ser, o de bajo coste, empleando fotogametr�a o c�maras infrarrojas.

Estos dispositivos obtienen una nube de puntos 3D (posici�n geom�trica), con informaci�n adicional arbitraria asociada, normalmente el color como m�nimo, aunque tambi�n frecuentemente otros datos como normales, temperatura, etc. Por las propias caracter�sticas que tiene el punto como primitiva gr�fica, respecto por ejemplo de los tradicionales pol�gonos habitualmente empleados en inform�tica gr�fica (no tienen �rea, no tienen orientaci�n, no est�n conectados, etc.), el render o visualizaci�n de nubes de puntos presenta una serie de retos si queremos obtener una visualizaci�n realista y de calidad.

Este proyecto se centra en la visualizaci�n avanzada de nubes de puntos, aplicando algunas de las m�s novedosas t�cnicas de render y haciendo uso de las caracter�sticas avanzadas de la API gr�fica multiplataforma OpenGL, lo que nos permite explotar el hardware de las tarjetas gr�ficas modernas. El resultado del proyecto ser� la implementaci�n de una herramienta multiplataforma para la visualizaci�n avanzada de nubes de puntos 3D.
