%
% CONCLUSIONES
%
\chapter[Conclusiones]{
	Conclusiones
	\label{nombre_referencia_al_capitulo_070}
}

En este cap�tulo se comentar� brevemente las conclusiones alcanzadas luego de la finalizaci�n de este proyecto fin de carrera. Continuando con las posibles lineas de trabajo que se podr�an seguir.

\section[Conclusiones]{Conclusiones}

Se podr�a decir que los siguientes objetivos fueron alcanzados:
\begin{itemize}
	\item Aprendizaje en el uso de una librer�a gr�fica como OpenGL.
	\item Programaci�n de programas para GPU con GLSL.
	\item Implementaci�n de un visualizador multiplataforma de modelos de puntos en tiempo real con el prop�sito de estudiar los diferentes algoritmos en \textit{GPU} que existen.
	\item Dise�o de herramientas para el an�lisis de resultados mediante lenguajes �giles.
\end{itemize}

Luego de la finalizaci�n de este proyecto, otras conclusiones alcanzadas fueron:
\begin{itemize}
	\item \textbf{El mismo tama�o para todos los puntos es un error:} El tiempo necesario para calcular como m�nimo la correcci�n de profundidad garantiza un mejora en la percepci�n visual adem�s de mejoras en el tiempo de renderizado para nubes de alta densidad. 
	
	\item \textbf{El aumento en la cantidad de puntos no siempre es mejor:} Al representar nubes con una muy alta densidad de puntos, implica un aumento desproporcionado del esfuerzo computacional necesario para acabar en muchos de los casos con resultados semejantes al renderizado con algoritmos simples. Incluso provocando que los algoritmos de \textit{blending} no tengan efecto al probocar superfices de solape �nfimas impidiendo que funcionen los algoritmos correctamente.
	
	\item \textbf{Investigar en sistemas de render que no usen normales:} A pesar de que los resultados obtenidos del render mediante algoritmos como \textit{deferred} son muy buenos. Las nubes de puntos en mucho de los \textit{datasets} carecen de este dato. Por lo que se exige de una precomputaci�n que una vez superando cierto tama�o empieza a no ser abordable, adem�s de que las normales obtenidas en muchos de los casos son erroneas probocando que las im�genes resultantes no se vean correctamente.
	
\end{itemize} 

\section[Posibles v�as de desarrollo]{Posibles v�as de desarrollo}

Luego de finalizar el desarrollo de este proyecto, surgieron diferentes ideas para continuar con el crecimiento de este proyecto:

\begin{itemize}
	\item \textbf{Rendimiento:} A pesar de que se intent� que el rendimiento obtenido del programa fuera lo m�s eficiente posible, habr�a que repasar y analizar bien las especificaciones del \textit{hardware} gr�fico para que las transacciones de datos fuera mas eficientes, intentando evitar sobrecomputaci�n evitable con buenas praxis.
 
	\item \textbf{Interfaz de usuario:} Hacer una migraci�n de \textit{GLFW} a \textit{Qt} con la idea de a�adir una interfaz de usuario y poder ofrecer m�s informaci�n y opciones al usuario.
	
	\item \textbf{Shaders:} Permitir al usuario el escribir sus propios shaders y probarlos online con los modelos cargados.
	
	\item \textbf{C�mara:} Mejorar el funcionamiento de la c�mara orbital, adem�s de a�adir mejoras en esta para poder moverse con m�s libertad por la escena.

	\item \textbf{Frustum culling:} Ahora mismo todos los puntos de la nube son enviados a GPU, ser�a interesante en desarrollar alg�n sistema para filtrar  estos datos mediante alg�n m�todo de \textit{hashing} espacial en funci�n de la c�mara antes de ser enviados para dibujar.
	
	\item \textbf{Order-independent transparency:} Implementar un sistema para intentar simplificar el \textit{blending} en tres pases a un sistema en un �nico pase que adem�s pueda ser aplicado a todos los algoritmos de rasterizaci�n.
	
\end{itemize}
%
% FIN DEL CAP�TULO
%